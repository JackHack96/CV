%% start of file `cv.tex'.
%% Copyright 2006-2015 Xavier Danaux (xdanaux@gmail.com), 2020-2022 moderncv maintainers (github.com/moderncv).
%
% This work may be distributed and/or modified under the
% conditions of the LaTeX Project Public License version 1.3c,
% available at http://www.latex-project.org/lppl/.


\documentclass[11pt,a4paper,sans]{moderncv}        % possible options include font size ('10pt', '11pt' and '12pt'), paper size ('a4paper', 'letterpaper', 'a5paper', 'legalpaper', 'executivepaper' and 'landscape') and font family ('sans' and 'roman')

% moderncv themes
\moderncvstyle{classic}                            % style options are 'casual' (default), 'classic', 'banking', 'oldstyle' and 'fancy'
\moderncvcolor{blue}                               % color options 'black', 'blue' (default), 'burgundy', 'green', 'grey', 'orange', 'purple' and 'red'
%\renewcommand{\familydefault}{\sfdefault}         % to set the default font; use '\sfdefault' for the default sans serif font, '\rmdefault' for the default roman one, or any tex font name
%\nopagenumbers{}                                  % uncomment to suppress automatic page numbering for CVs longer than one page

% adjust the page margins
\usepackage[scale=0.75]{geometry}
\setlength{\footskip}{149.60005pt}                 % depending on the amount of information in the footer, you need to change this value. comment this line out and set it to the size given in the warning
%\setlength{\hintscolumnwidth}{3cm}                % if you want to change the width of the column with the dates
%\setlength{\makecvheadnamewidth}{10cm}            % for the 'classic' style, if you want to force the width allocated to your name and avoid line breaks. be careful though, the length is normally calculated to avoid any overlap with your personal info; use this at your own typographical risks...

% font loading
% for luatex and xetex, do not use inputenc and fontenc
% see https://tex.stackexchange.com/a/496643
\ifxetexorluatex
  \usepackage{fontspec}
  \usepackage{unicode-math}
  \defaultfontfeatures{Ligatures=TeX}
  \setmainfont{Latin Modern Roman}
  \setsansfont{Latin Modern Sans}
  \setmonofont{Latin Modern Mono}
  \setmathfont{Latin Modern Math}
\else
  \usepackage[utf8]{inputenc}
  \usepackage[T1]{fontenc}
  \usepackage{lmodern}
\fi

% document language
\usepackage[italian]{babel}  % FIXME: using spanish breaks moderncv

% personal data
\name{Matteo}{Iervasi}
%\title{Curriculum Vitae}
%\born{27 January 1996}
%\address{Via Olmo 20A}{37068}{VR}
\email{matteoiervasi@gmail.com}
\homepage{https://jackhack96.github.io/}
\phone[mobile]{+39~377~275~9725}

% Social icons
\social[linkedin]{matteo-iervasi}

%\social[github]{JackHack96}
%\social[gitlab]{JackHack96}
%\social[stackoverflow]{3760726/matteo-iervasi}

\photo[64pt][0.4pt]{picture}                       % optional, remove / comment the line if not wanted; '64pt' is the height the picture must be resized to, 0.4pt is the thickness of the frame around it (put it to 0pt for no frame) and 'picture' is the name of the picture file

% bibliography adjustments (only useful if you make citations in your resume, or print a list of publications using BibTeX)
%   to show numerical labels in the bibliography (default is to show no labels)
%\makeatletter\renewcommand*{\bibliographyitemlabel}{\@biblabel{\arabic{enumiv}}}\makeatother
\renewcommand*{\bibliographyitemlabel}{[\arabic{enumiv}]}
%   to redefine the bibliography heading string ("Publications")
%\renewcommand{\refname}{Articles}

% bibliography with mutiple entries
%\usepackage{multibib}
%\newcites{book,misc}{{Books},{Others}}
%----------------------------------------------------------------------------------
%            content
%----------------------------------------------------------------------------------
\begin{document}
%\begin{CJK*}{UTF8}{gbsn}                          % to typeset your resume in Chinese using CJK
%-----       resume       ---------------------------------------------------------
\makecvtitle

Ingegnere informatico specializzato nello sviluppo firmware con una solida esperienza nella programmazione a basso livello e nello sviluppo di progetti basati su microcontrollore, sempre alla ricerca di opportunità per applicare la mia competenza tecnica e contribuire a progetti innovativi e di impatto. Esperto nella creazione di soluzioni firmware efficienti e affidabili per una vasta gamma di sistemi embedded. 
Conoscenza approfondita del linguaggio C, C++ e Python e di molti ambienti di sviluppo embedded. Esperto anche nell'utilizzo del progetto Yocto per la creazione di distribuzioni Linux embedded.

\section{Formazione}
% arguments 3 to 6 can be left empty
\cventry{2018--2020}{Laurea magistrale in Ingegneria e Scienze Informatiche}{Università degli Studi di Verona}{Verona}{}{
	Corsi rilevanti:
	\begin{itemize}
		\item Progettazione di sistemi embedded
		\item Sistemi embedded di rete
		\item Fisica dei dispositivi integrati
		\item Sistemi dinamici
	\end{itemize}
}
\cventry{2015--2018}{Laurea triennale in Informatica}{Università degli Studi di Verona}{Verona}{}{
	Corsi rilevanti:
	\begin{itemize}
		\item Sistemi operativi
		\item Ingegneria del software
		\item Teoria dei sistemi ed Elaborazione di segnali e immagini
		\item Linguaggi e compilatori
	\end{itemize}
}
\cventry{2010--2015}{Diploma superiore}{Liceo Scientifico Angelo Messedaglia}{Verona}{}{
	Frequentato il curriculum ``\textit{Scienze Applicate}'', incentrato su Informatica, Fisica, Chimica e Biologia.
	Obiettivi rilevanti:
	\begin{itemize}
		\item Sviluppato un software per il controllo dello spettrofotometro del laboratorio di chimica
	\end{itemize}
}

%\section{Master thesis}
%\cvitem{Title}{\emph{Integrating synthetic and real components of a cyber-physical production system}}
%\cvitem{Supervisor}{Prof. Franco Fummi}
%\cvitem{Advisor}{Dr. Stefano Centomo}
%\cvitem{Abstract}{
%	One of the key aspects of \emph{Industry 4.0} is the concept of \texttt{Digital Twins}, as they are an enabling
%	technology for things like predictive maintenance, real-time production optimization, on-demand product
%	customization and so on\ldots
%	A limiting factor in the creation of \emph{Digital Twins} is the abundance of incompatible modeling languages. Among
%	the research and the projects that tries to overcome this issue, \texttt{AutomationML} is increasingly cited and used
%	as a vendor-neutral language for model exchange.
%	This work proposes a simple direct approach for the integration of models in CPPS systems, using \texttt{AutomationML}
%	as the base technology.
%}

\section{Esperienza}
\cventry{Set 2020-- Oggi}{Ingegnere dei sistemi embedded}{EDALab S.r.l.}{San Giovanni Lupatoto}{}{Sviluppo di firmware e software embedded per terze parti.
\begin{itemize}
	\item Collaborazione con i clienti per la creazione e lo sviluppo di soluzioni firmware per diversi sistemi embedded
	\item Implementazione di software low-level, ottimizzazione delle performance e test di compatibilità hardware
	\item Sviluppo di sistemi real-time per applicazioni industriali
\end{itemize}
}

\cventry{Ago 2020}{Tirocinio}{EDALab S.r.l.}{San Giovanni Lupatoto}{}{Integrazione di un sistema affidabile di aggiornamento basato su SWUpdate per piattaforma BoxIO.}
\cventry{Dic 2019}{Tirocinio}{Università degli Studi di Verona}{Verona}{}{Sviluppo dell'immagine di un sistema operativo per i monitor delle aule basati su Raspberry Pi.}
\cventry{Gen 2017-- Mar 2018}{Tirocinio}{Sordato S.r.l.}{Monteforte d'Alpone}{}{Sviluppo di un sistema di controllo automatico per macchine di fermentazione del vino.}
\cventry{2017--2018}{Assistente professore}{Università degli Studi di Verona}{Verona}{}{Ho lavorato come assistente professore nei seguenti corsi:
\begin{itemize}
	\item Sistemi operativi
	\item Programmazione I
	\item Programmazione II
\end{itemize}
}

\subsection{Volontariato}
\cventry{2017-- Oggi}{Tecnico}{AVIS}{Vigasio}{}{Sono volontario presso l'associazione AVIS locale, come tecnico informatico.}

\section{Lingue}
\cvitemwithcomment{Italiano}{Lingua madre}{}
\cvitemwithcomment{Inglese}{Livello professionale}{}

\section{Capacità}
\cvitem{Linguaggi di programmazione}{Ottima conoscenza di C, C++ e Python.}
\cvitem{Sviluppo firmware}{Esperto nella programmazione low-level per svariati microcontrollori con vari ambienti di lavoro e compilatori, es. IAR®, GCC/Clang e Keil®.}
\cvitem{MCUs}{Esperto nello sviluppo di soluzioni firmware per le piattaforme ST® STM32, Renesas® RL78, Renesas® RX130, Renesas® RA, NXP® Kinetis, Cypress® FM4 e altri microcontrollori basati su ARM, 
	Microchip® PIC18, PIC24 e PIC32, Intel® 8051 ed Espressif® ESP32/ESP8266.}
\cvitem{Sistemi embedded}{Ottima conoscenza del progetto Yocto per la creazione di distribuzioni Linux embedded persoanlizzate e di Qt/QML per la creazione di complesse interfacce uomo-macchina.}
\cvitem{Sistemi operativi}{Buona conoscenza nello sviluppo generale per sistemi GNU/Linux e Microsoft® Windows®.}
%\cvitem{Office Automation}{Skilled in using the Microsoft® Office® suite and LibreOffice.}
%\cvitem{Web technologies}{HTML, CSS, Familiar with WordPress, Joomla}
\cvitem{Scripting}{Esperto in automazione del sistema operativo tramite il linguaggio di scripting Bash e Python.}


%\section{Computer Science skills}
% \cvitem{Skill matrix}{Alternatively, provide a skill matrix to show off your skills}
%% Skill matrix as an alternative to rate one's skills, computer or other.

%% Adjusts width of skill matrix columns.
%% Usage \setcvskillcolumns[<width>][<factor>][<exp_width>]
%% <width>, <exp_width> should be lengths smaller than \textwidth, <factor> needs to be between 0 and 1.
%% Examples:
% \setcvskillcolumns[5em][][]%    adjust first column. Same as \setcvskillcolumns[5em]
% \setcvskillcolumns[][0.45][]%   adjust third (skill) column. Same as \setcvskillcolumns[][0.45]
% \setcvskillcolumns[][][\widthof{``Year''}]%     adjust fourth (years) column.
% \setcvskillcolumns[][0.45][\widthof{``Year''}]%
% \setcvskillcolumns[\widthof{``Languag''}][0.48][]
% \setcvskillcolumns[\widthof{``Languag''}]%

%% Adjusts width of legend columns. Usage \setcvskilllegendcolumns[<width>][<factor>]
%% <factor> needs to be between 0 and 1. <width> should be a length smaller than \textwidth
%% Examples:
% \setcvskilllegendcolumns[][0.45]
% \setcvskilllegendcolumns[\widthof{``Legend''}][0.45]
% \setcvskilllegendcolumns[0ex][0.46]% this is usefull for the banking style

%% Add a legend if you are using \cvskill{<1-5>} command or \cvskillentry
%% Usage \cvskilllegend[*][<post_padding>][<first_level>][<second_level>][<third_level>][<fourth_level>][<fifth_level>]{<name>}
% \cvskilllegend % insert default legend without lines
%\cvskilllegend*[1em]{}% adjust post spacing
% \cvskilllegend*{Legend}%  Alternatively add a description string
%% adjust the legend entries for other languages, here German
% \cvskilllegend[0.2em][Grundkenntnisse][Grundkenntnisse und eigene Erfahrung in Projekten][Umfangreiche Erfahrung in Projekten][Vertiefte Expertenkenntnisse][Experte\,/\,Spezialist]{Legende}

%% Alternative legend style with the first three skill levels in one column
%% Usage \cvskillplainlegend[*][<post_padding>][<first_level>][<second_level>][<third_level>][<fourth_level>][<fifth_level>]{<name>}
% \setcvskilllegendcolumns[][0.6]%  works for classic, casual, banking
% \setcvskilllegendcolumns[][0.55]%  works better for oldstyle and fancy
% \cvskillplainlegend{}
% \cvskillplainlegend[0.2em][Grundkenntnisse][Grundkenntnisse und eigene Erfahrung in Projekten][Umfangreiche Erfahrung in Projekten][Vertiefte Expertenkenntnisse][Experte/Guru]{Legende}

%% Add a head of the skill matrix table with descriptions.
%% Usage \cvskillhead[<post_padding>][<Level>][<Skill>][<Years>][<Comment>]%
%\cvskillhead[-0.1em]%   this inserts the standard legend in english and adjust padding
%% Adjust head of the skill matrix for other languages
% \cvskillhead[0.25em][Level][F\"ahigkeit][Jahre][Bemerkung]

%% \cvskillentry[*][<post_padding>]{<skill_cathegory>}{<0-5>}{<skill_name>}{<years_of_experience>}{<comment>}%
%% Example usages:
%\cvskillentry*{Language:}{5}{C}{\number\numexpr\year-2010\relax~}{As a firmware engineer, C is often my only language available. I also know compiler-specific extension for GCC and IAR®.}
%\cvskillentry{}{4}{C++}{\number\numexpr\year-2018\relax~}{Though not as good as I'm with C, I use C++ often, generally for HMI in embedded projects with frameworks such as Qt.}
%\cvskillentry{}{3}{Python}{\number\numexpr\year-2018\relax~}{I developed some personal projects with Python and it's my go-to choice when doing complex scripts.}
%\cvskillentry{}{2}{Java}{\number\numexpr\year-2014\relax~}{I've used Java in many personal projects and also at the university, but I don't generally like to use it.}
%\cvskillentry{}{3}{\LaTeX}{\number\numexpr\year-2018\relax~}{I've used \LaTeX for all my documents at the University and I still use it sometimes for complex documents, as I generally prefer it over Microsoft® Office®.}
%\cvskillentry*{OS:}{4}{GNU/Linux}{\number\numexpr\year-2010\relax~}{I've used many GNU/Linux distributions in the past and I've settled on Ubuntu for a decade now.}% notice the use of the starred command and the optional
%\cvskillentry{}{4}{Microsoft® Windows®}{\number\numexpr\year-2005\relax~}{I can use any Microsoft® Windows® version though I prefer to use Linux when possible.}
%\cvskillentry*{MCU:}{4}{ST® STM32}{\number\numexpr\year-2018\relax~}{ST has been my main developing platform for quite some time.}
%\cvskillentry{}{4}{Renesas® RL78}{\number\numexpr\year-2020\relax~}{I've worked with this 16-bit architecture for a variety of customers.}
%\cvskillentry{}{3}{Renesas® RX130}{\number\numexpr\year-2020\relax~}{My knowledge on this platform is more limited than RL78 because of fewer projects, but still good.}
%\cvskillentry{}{2}{Intel® 8051}{\number\numexpr\year-2020\relax~}{I've used this architecture on some projects with success.}
%\cvskillentry{}{3}{Other ARM-based}{\number\numexpr\year-2020\relax~}{I've worked with many different ARM-based architectures apart from ST, like Renesas® RA, NXP® Kinetis, Cypress® FM4 and others.}
%\cvskillentry{}{5}{Arduino}{\number\numexpr\year-2018\relax~}{Arduino was by starting point in the embedded world. It's still good for rapid prototyping.}
%\cvskillentry*{Others:}{4}{Yocto}{\number\numexpr\year-2020\relax~}{Experienced using Yocto Project for customizing embedded Linux distributions.}
%\cvskillentry{}{4}{Git}{\number\numexpr\year-2014\relax~}{I use Git since .}
%\cvskillentry*[1em]{Methods}{4}{SCRUM}{8}{SCRUM master for 5 years}
%% \cvskill{<0-5>} command
% \cvitem{\textbackslash{cvskill}:}{Skills can be visually expressed by the \textbackslash{cvskill} command, e.g. \cvskill{2}}

\section{Interessi e hobby}
\cvitem{Elettronica}{Mi interesso di elettronica, dalla progettazione di circuiti alla creazione dei PCB.}
\cvitem{Stampa 3D}{Seguo il mondo della stampa 3D e mi piace progettare su CAD oggetti che poi utilizzo nei miei progetti.}
\cvitem{DIY}{Mi piace arrangiarmi nelle varie riparazioni, da elettrodomestici ad arredamenti vari.}



\clearpage

%\clearpage\end{CJK*}                              % if you are typesetting your resume in Chinese using CJK; the \clearpage is required for fancyhdr to work correctly with CJK, though it kills the page numbering by making \lastpage undefined
\end{document}


%% end of file `template.tex'.